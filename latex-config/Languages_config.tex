% updated 2026-01-03
% Language and font configuration 语言和字体配置 言語とフォントの設定

% region ==== 字体设置 ==== フォント設定 ====
% 基于 fontspec 和 xeCJK 宏包
% 实现英语, 日语, 中文混排  
% 英語, 日本語, 中国語 の混在を実現
%
% —— 欧文字体 —— 英語フォント —— 
% 设置英文的主字体、无衬线字体和等宽字体
% mainfont: 主字体 (serif)
% sansfont: 无衬线字体 (sans-serif)
% monofont: 等宽字体 (monospace)
%—— 日本語字体 —— 日本語フォント —— 
% setlength: \parindent 设置段落缩进 段落冒頭のインデント
% indentfirst: 
%     开启section后第一段的段首空格 section後の最初の段落もインデント 
%     注意:正式论文不一定需要, 可根据需要选择是否使用 注意: 正式な論文では必ずしも必要ではないので, 必要に応じて使用するかどうかを選択可能
% package xeCJK: 支持中日韩字体 日中韓フォントサポート
% xeCJKsetup: auto fallback 优先加载日语, 没有的字体则自动回退到中文字体 多言語混在の自動フォントフォールバック機能を有効化
%—— 中文字体 —— 中国語フォント —— 
%—— 纯中文环境 ——


% 检查宏包是否存在模版
%\IfFileExists{ceshi.sty}{%
%    \typeout{ ceshi 宏包已安装}%
%}{%
%    \typeout{PackageError 指令测试}%
%    \PackageError{宏包缺失}{%
%         ceshi.sty 未找到%
%    }{%
%        请安装 ceshi 宏包,或联系模板维护者。\MessageBreak % chktex 1
%    }
%}

%依赖检查
% 检查宏包是否存在
\IfFileExists{fontspec.sty}{%
    \typeout{ fontspec 宏包已安装}%
}{%
    \PackageError{宏包缺失}{%
         fontspec.sty 未找到%
    }{%
        请安装 fontspec 宏包,或联系模板维护者。\MessageBreak % chktex 1
    }
}
\IfFileExists{xeCJK.sty}{%
    \typeout{ xeCJK 宏包已安装}%
}{%
    \PackageError{宏包缺失}{%
         xeCJK.sty 未找到%
    }{%
        请安装 xeCJK 宏包,或联系模板维护者。\MessageBreak % chktex 1
    }
}
\IfFileExists{indentfirst.sty}{%
    \typeout{ indentfirst 宏包已安装}%
}{%
    \PackageError{宏包缺失}{%
         indentfirst.sty 未找到%
    }{%
        请安装 indentfirst 宏包,或联系模板维护者。\MessageBreak % chktex 1
    }
}

% 英语 / 欧文
\usepackage{fontspec}
\setmainfont{Times New Roman}
\setsansfont{Helvetica}
\setmonofont{Courier New}

% 日语 / 日本語
\setlength{\parindent}{10.5pt} % 设置段落缩进
\usepackage{indentfirst} 
\usepackage{xeCJK} 
\punctstyle{quanjiao}
%草稿: 进阶写法 \usepackage[AutoFakeBold=true, AutoFakeSlant=true]{xeCJK}
\xeCJKsetup{AutoFallBack=true}
\setCJKmainfont[BoldFont=YuMincho Demibold]{YuMincho Medium}
\setCJKsansfont{Hiragino Sans W5}
\setCJKmonofont{Hiragino Sans W3}

% 中文 / 中国語
\setCJKfallbackfamilyfont{\CJKrmdefault}{Songti SC} 
\setCJKfallbackfamilyfont{\CJKsfdefault}{Heiti SC} 
\setCJKfallbackfamilyfont{\CJKttdefault}{PingFang SC}
% 定义中文字体族
\setCJKfamilyfont{CNfont_songti}[BoldFont=Songti SC Black]{Songti SC}
\setCJKfamilyfont{CNfont_heiti}{Heiti SC}
\setCJKfamilyfont{CNfont_pingfangti}{PingFang SC}
\NewDocumentEnvironment{Chinese}{}{%
    \xeCJKsetup{AutoFallBack=true}%
    \CJKfamily{CNfont_songti}
}{}
% endregion

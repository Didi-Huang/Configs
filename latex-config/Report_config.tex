% update 2026-01-17


%\documentclass[10.5pt]{extarticle}

\setcounter{secnumdepth}{5}
\usepackage{titlesec}
\usepackage{fvextra}
\usepackage{array}
\usepackage[super,numbers]{natbib} % 参考文献样式
\usepackage{geometry}
\geometry{a4paper, left=2.5cm, right=2.5cm, top=2.5cm, bottom=2.5cm}
\usepackage{pdfpages} % 导入pdf作为封面

\usepackage{float}
\usepackage{enumitem}
\usepackage{url}
\usepackage{pgfplots} % 支持绘图
\pgfplotsset{compat=1.18} % 设置 pgfplots 的兼容性版本

% updated 2026-01-03
% Language and font configuration 语言和字体配置 言語とフォントの設定

% region ==== 字体设置 ==== フォント設定 ====
% 基于 fontspec 和 xeCJK 宏包
% 实现英语, 日语, 中文混排  
% 英語, 日本語, 中国語 の混在を実現
%
% —— 欧文字体 —— 英語フォント —— 
% 设置英文的主字体、无衬线字体和等宽字体
% mainfont: 主字体 (serif)
% sansfont: 无衬线字体 (sans-serif)
% monofont: 等宽字体 (monospace)
%—— 日本語字体 —— 日本語フォント —— 
% setlength: \parindent 设置段落缩进 段落冒頭のインデント
% indentfirst: 
%     开启section后第一段的段首空格 section後の最初の段落もインデント 
%     注意:正式论文不一定需要, 可根据需要选择是否使用 注意: 正式な論文では必ずしも必要ではないので, 必要に応じて使用するかどうかを選択可能
% package xeCJK: 支持中日韩字体 日中韓フォントサポート
% xeCJKsetup: auto fallback 优先加载日语, 没有的字体则自动回退到中文字体 多言語混在の自動フォントフォールバック機能を有効化
%—— 中文字体 —— 中国語フォント —— 
%—— 纯中文环境 ——


% 检查宏包是否存在模版
%\IfFileExists{ceshi.sty}{%
%    \typeout{ ceshi 宏包已安装}%
%}{%
%    \typeout{PackageError 指令测试}%
%    \PackageError{宏包缺失}{%
%         ceshi.sty 未找到%
%    }{%
%        请安装 ceshi 宏包,或联系模板维护者。\MessageBreak % chktex 1
%    }
%}

%依赖检查
% 检查宏包是否存在
\IfFileExists{fontspec.sty}{%
    \typeout{ fontspec 宏包已安装}%
}{%
    \PackageError{宏包缺失}{%
         fontspec.sty 未找到%
    }{%
        请安装 fontspec 宏包,或联系模板维护者。\MessageBreak % chktex 1
    }
}
\IfFileExists{xeCJK.sty}{%
    \typeout{ xeCJK 宏包已安装}%
}{%
    \PackageError{宏包缺失}{%
         xeCJK.sty 未找到%
    }{%
        请安装 xeCJK 宏包,或联系模板维护者。\MessageBreak % chktex 1
    }
}
\IfFileExists{indentfirst.sty}{%
    \typeout{ indentfirst 宏包已安装}%
}{%
    \PackageError{宏包缺失}{%
         indentfirst.sty 未找到%
    }{%
        请安装 indentfirst 宏包,或联系模板维护者。\MessageBreak % chktex 1
    }
}

% 英语 / 欧文
\usepackage{fontspec}
\setmainfont{Times New Roman}
\setsansfont{Helvetica}
\setmonofont{Courier New}

% 日语 / 日本語
\setlength{\parindent}{10.5pt} % 设置段落缩进
\usepackage{indentfirst} 
\usepackage{xeCJK} 
\punctstyle{quanjiao}
%草稿: 进阶写法 \usepackage[AutoFakeBold=true, AutoFakeSlant=true]{xeCJK}
\xeCJKsetup{AutoFallBack=true}
\setCJKmainfont[BoldFont=YuMincho Demibold]{YuMincho Medium}
\setCJKsansfont{Hiragino Sans W5}
\setCJKmonofont{Hiragino Sans W3}

% 中文 / 中国語
\setCJKfallbackfamilyfont{\CJKrmdefault}{Songti SC} 
\setCJKfallbackfamilyfont{\CJKsfdefault}{Heiti SC} 
\setCJKfallbackfamilyfont{\CJKttdefault}{PingFang SC}
% 定义中文字体族
\setCJKfamilyfont{CNfont_songti}[BoldFont=Songti SC Black]{Songti SC}
\setCJKfamilyfont{CNfont_heiti}{Heiti SC}
\setCJKfamilyfont{CNfont_pingfangti}{PingFang SC}
\NewDocumentEnvironment{Chinese}{}{%
    \xeCJKsetup{AutoFallBack=true}%
    \CJKfamily{CNfont_songti}
}{}
% endregion
  %语言配置

% updated 2026-01-03
% Math configuration 数学配置 数学の設定

% region ==== 数学字体设置 ==== 数学フォント設定 ====
% 基于 mathtools 和 unicode-math 宏包
% 实现数学符号和字体的配置
% 数学記号とフォントの設定を実現    
%—— 数学扩展宏包 —— 数学拡張パッケージ —— 
% mathtools: 数学扩展宏包, 提供更多数学符号和环境 数学拡張パッケージ, より多くの数学記号と環境を提供
% unicode-math: 支持 Unicode 数学字体的宏包 Unicode 数学フォントをサポートするパッケージ
% setmathfont: 设置数学字体
% Symbol: 学校指定 Symbol 字体 用于数字和直立希腊字母 学校指定の Symbol フォントを使用して数字と直立ギリシャ文字を表示
%—— 物理学符号宏包 —— 物理学記号パッケージ —— 
% physics: 提供物理学相关的符号和命令 物理学に関連する記号とコマンドを提供  
%依赖检查
% 检查宏包是否存在
\IfFileExists{mathtools.sty}{%
    \typeout{ mathtools 宏包已安装}%
}{%
    \PackageError{宏包缺失}{%
         mathtools.sty 未找到%
    }{%
        请安装 mathtools 宏包,或联系模板维护者。\MessageBreak % chktex 1
    }
}
\IfFileExists{unicode-math.sty}{%
    \typeout{ unicode-math 宏包已安装}%
}{%
    \PackageError{宏包缺失}{%
         unicode-math.sty 未找到%
    }{%
        请安装 unicode-math 宏包,或联系模板维护者。\MessageBreak % chktex 1
    }
}
\IfFileExists{physics.sty}{%
    \typeout{ physics 宏包已安装}%
}{%
    \PackageError{宏包缺失}{%
         physics.sty 未找到%
    }{%
        请安装 physics 宏包,或联系模板维护者。\MessageBreak % chktex 1
    }
}
\usepackage{mathtools} % 数学扩展宏包 要在 unicode-math 之前加载
\usepackage{unicode-math}
\setmathfont{NewComputerModernMath-Regular}  % 主数学字体
\setmathfont{Symbol}[ % 学校指定Symbol字体用于直立希腊字母
    range=up/{greek,Greek},       % 直立希腊字母
    Scale=MatchLowercase          % 保持大小匹配
]
\renewcommand{\familydefault}{\rmdefault}  % 强制使用衬线字体
\usepackage{physics} % 物理学符号宏包
% endregion
       %数学公式配置

\usepackage{graphicx} % 支持插入图片
\usepackage{pgf} % 支持 pgf 格式图片
% 设置 PGF 使用衬线字体

\usepackage{hyperref} % 支持超链接
\pdfstringdefDisableCommands{
  \def\\{ }  % 在PDF书签中将\\替换为空格
  \def\hs{ } % 在PDF书签中忽略\hs
}
\hypersetup{colorlinks=true, linkcolor=black, citecolor=black, urlcolor=black}





\usepackage{multicol} % 支持多栏布局
\usepackage{textcomp}
\usepackage{comment}

% updated 2026-01-03
% Table configuration 表格配置 表の設定
%=== 表格支持 === 表サポート ===
%依赖检查
% 检查宏包是否存在
\IfFileExists{booktabs.sty}{%
    \typeout{ booktabs宏包已安装}%
}{%
    \PackageError{宏包缺失}{%
         booktabs.sty 未找到%
    }{%
        请安装 booktabs 宏包,或联系模板维护者。\MessageBreak % chktex 1
    }
}
\IfFileExists{threeparttable.sty}{%
    \typeout{ threeparttable宏包已安装}%
}{%
    \PackageError{宏包缺失}{%
         threeparttable.sty 未找到%
    }{%
        请安装 threeparttable 宏包,或联系模板维护者。\MessageBreak % chktex 1
    }
}
\IfFileExists{multirow.sty}{%
    \typeout{ multirow宏包已安装}%
}{%
    \PackageError{宏包缺失}{%
         multirow.sty 未找到%
    }{%
        请安装 multirow 宏包,或联系模板维护者。\MessageBreak % chktex 1
    }
}
\IfFileExists{longtable.sty}{%
    \typeout{ longtable宏包已安装}%
}{%
    \PackageError{宏包缺失}{%
         longtable.sty 未找到%
    }{%
        请安装 longtable 宏包,或联系模板维护者。\MessageBreak % chktex 1
    }
}
%加载宏包
\usepackage{booktabs} % 支持表格
\usepackage{threeparttable} % 支持表格注释
\usepackage{multirow} % 支持表格多行合并
\usepackage{longtable} % 支持跨页表格
\usepackage{array} % 支持自定义列宽和对齐方式 %表格配置





\usepackage{fancyhdr} % 支持页眉页脚
\usepackage{lipsum} % 用于生成示例文本
\usepackage{listings}
\renewcommand{\lstlistingname}{コード}
\usepackage[newfloat]{minted} % 支持代码高亮
\setminted{
    style=xcode,   % friendly / xcode / default 可以试试
    fontsize=\footnotesize,
    breaklines,
    linenos,
    numbersep=5pt,
    autogobble,
    frame=single
}

\usepackage{annotate-equations} % 支持公式注释
\usepackage[dvipsnames]{xcolor}

%---- Tikz 环境 ---- Tikzパッケージ ----
\usepackage{tikz}
\def\mathdefault#1{#1}
\usetikzlibrary{arrows.meta} % 使用 TikZ 的箭头库
\usetikzlibrary{positioning}
\usetikzlibrary{angles, quotes}
\usetikzlibrary{shapes, arrows, positioning}
\usepackage{circuitikz} 
\ctikzset{resistors/thickness=1}
\ctikzset{grounds/thickness=1}


% --- 定义图层 ---
\pgfdeclarelayer{background}
\pgfdeclarelayer{foreground}
\pgfsetlayers{background,main,foreground}
% --- 定义环境 overlayimage ---
\newenvironment{overlayimage}[2][]{%
  \begin{tikzpicture}[#1]
    % 下层:背景输入的 PGF 图
    \begin{pgfonlayer}{background}
      \input{#2}%
    \end{pgfonlayer}
}{%
  \end{tikzpicture}
}

\usepackage{xspace}
\newcommand{\hs}{\hspace{0.8em}}  % 文本用 half-space
\newcommand{\hd}{\hspace{0.5em}}  % 文本用 double-space
%\NewDocumentCommand{\hl}{ O{yellow} m }{%高级命令
%  \colorbox{#1}{#2}%
%}









%=== 参考文献与格式设置 === 参考文献とフォーマット設定 ===
\renewcommand{\refname}{参考文献}   % 适用于 report, book
\renewcommand{\bibname}{参考文献}    % 适用于 bibtex
\bibliographystyle{plain}
\renewcommand{\baselinestretch}{1.0}% % 设置行距为1.0倍

\titleformat{\section}
    {\bfseries\fontsize{12}{12}\selectfont} % 12pt标题,\hspace{0.8em}加粗 行距
    {\thesection}{0.3em}{\raggedright}% 缩小编号和标题之间的间距
\titleformat{\subsection}
    {\bfseries\fontsize{11}{11}\selectfont} % 11pt标题,\hspace{0.8em}加粗 行距
    {\thesubsection}{0.3em}{\raggedright} % 缩小编号和标题之间的间距
\titleformat{\subsubsection}
    {\bfseries\fontsize{11}{11}\selectfont} % 11pt标题,\hspace{0.8em}加粗 行距
    {\thesubsubsection}{0.3em}{\raggedright} % 缩小编号和标题之间的间距 %第二个是让标题对齐左边

\renewcommand\normalsize{\fontsize{10.5}{15}\selectfont}% % 正文字体大小为10.5pt,\hspace{0.8em}行距为10.5pt
\renewcommand{\thesection}{\arabic{section}.} % 章节编号加点
\renewcommand{\thesubsection}{\thesection\arabic{subsection}} % 避免多余的点

% update 2026-01-03
% Caption configuration 标题配置 キャプションの設定
% region === 图表标题设置 === 図表キャプション設定 ===
\usepackage{caption}
\DeclareCaptionFont{ninept}{\fontsize{9}{9}\selectfont}%图表标题字体大小为9pt
\captionsetup[figure]
{
  name=図,%or name=Figure,
  labelsep=period,
  font= ninept,
  labelfont={bf},
  textfont={normalfont}
}
\captionsetup[table]
{
  name=表,%or name=Table,
  labelsep=period,
  font= ninept,
  labelfont={bf},
  textfont={normalfont}
}
% endregion  %图表标题配置

\newcommand{\insertpgf}[6][0.5]{ 
    \begin{figure}[H] 
        #5 
        \centering 
        \resizebox{#1\textwidth}{!}{\input{#2}
        #6
        } 
        \caption{#3}\label{#4} 
    \end{figure}
}
\newcommand{\insertpgfx}[5][0.5]{ 
    \begin{figure}[H] 
        #4 
        \centering 
        \resizebox{#1\textwidth}{!}{\input{#2}
        #5
        }\label{#3} 
    \end{figure}
}




%用法:\insertpgf{発表/実験1.pgf}{実験1:各波長の$I$-$V$特性}{fig:exp1}{\vspace{-3em}}
\newcommand{\insertpng}[5][0.5]{
    \begin{figure}[H]
        #5
        \centering
        \includegraphics[width=#1\textwidth]{#2}
        \caption{#3}\label{#4}
    \end{figure}
}

\newcommand{\insertpdf}[5][0.5]{
    \begin{figure}[H]
        #5
        \centering
        \includegraphics[width=#1\textwidth]{#2}
        \caption{#3}\label{#4}
    \end{figure}
}

\newcommand{\insertpair}[6][0.45]{%
    \begin{figure}[H]
        \centering
        \begin{minipage}{0.49\textwidth}
            \centering
            \pgfimg[#1]{#2}{#3}
        \end{minipage}\hfill
        \begin{minipage}{0.49\textwidth}
            \centering
            \pgfimg[#1]{#4}{#5}
        \end{minipage}
        \caption{#6}\label{fig:pair}
    \end{figure}
}
\newcommand{\NaturalElementTextFormat}[4]
{
  \begin{minipage}{2.3cm}
    \centering
      \textbf{#1}
      \linebreak \linebreak
      {\Huge \textbf{{#3}}}
      \linebreak
       {#4}
      \linebreak
      {\small {#2}} 
  \end{minipage}
}


%
%{\LARGE\bfseries 連続体力学期末レポート \par}
%\vskip 1em
%{\large 黄宏迪 \par}
%{\large 1224038 \par}
%\vskip 1em
%{\large \the\year 年\the\month 月\the\day 日 \par}



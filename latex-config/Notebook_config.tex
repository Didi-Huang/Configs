% updated 2026-01-09
% Notebook configuration 笔记本配置 ノートブックの設定
% 现在现有的配置是适合documentclass article的,之后要改为notebook或者book类
% 量子力学笔记本目前的documentclass是extarticle
% 之后还要继续把 颜色配置等分离出去
% 目前的配置文件结构如下:

% 结构:



%\documentclass[10.5pt]{article}
\usepackage{multicol}
\setlength{\columnsep}{20pt} % 两栏之间的间距
\setlength{\columnseprule}{0.4pt} % 中间竖线宽度,默认是 0 表示无


% updated 2026-01-03
% Language and font configuration 语言和字体配置 言語とフォントの設定

% region ==== 字体设置 ==== フォント設定 ====
% 基于 fontspec 和 xeCJK 宏包
% 实现英语, 日语, 中文混排  
% 英語, 日本語, 中国語 の混在を実現
%
% —— 欧文字体 —— 英語フォント —— 
% 设置英文的主字体、无衬线字体和等宽字体
% mainfont: 主字体 (serif)
% sansfont: 无衬线字体 (sans-serif)
% monofont: 等宽字体 (monospace)
%—— 日本語字体 —— 日本語フォント —— 
% setlength: \parindent 设置段落缩进 段落冒頭のインデント
% indentfirst: 
%     开启section后第一段的段首空格 section後の最初の段落もインデント 
%     注意:正式论文不一定需要, 可根据需要选择是否使用 注意: 正式な論文では必ずしも必要ではないので, 必要に応じて使用するかどうかを選択可能
% package xeCJK: 支持中日韩字体 日中韓フォントサポート
% xeCJKsetup: auto fallback 优先加载日语, 没有的字体则自动回退到中文字体 多言語混在の自動フォントフォールバック機能を有効化
%—— 中文字体 —— 中国語フォント —— 
%—— 纯中文环境 ——


% 检查宏包是否存在模版
%\IfFileExists{ceshi.sty}{%
%    \typeout{ ceshi 宏包已安装}%
%}{%
%    \typeout{PackageError 指令测试}%
%    \PackageError{宏包缺失}{%
%         ceshi.sty 未找到%
%    }{%
%        请安装 ceshi 宏包,或联系模板维护者。\MessageBreak % chktex 1
%    }
%}

%依赖检查
% 检查宏包是否存在
\IfFileExists{fontspec.sty}{%
    \typeout{ fontspec 宏包已安装}%
}{%
    \PackageError{宏包缺失}{%
         fontspec.sty 未找到%
    }{%
        请安装 fontspec 宏包,或联系模板维护者。\MessageBreak % chktex 1
    }
}
\IfFileExists{xeCJK.sty}{%
    \typeout{ xeCJK 宏包已安装}%
}{%
    \PackageError{宏包缺失}{%
         xeCJK.sty 未找到%
    }{%
        请安装 xeCJK 宏包,或联系模板维护者。\MessageBreak % chktex 1
    }
}
\IfFileExists{indentfirst.sty}{%
    \typeout{ indentfirst 宏包已安装}%
}{%
    \PackageError{宏包缺失}{%
         indentfirst.sty 未找到%
    }{%
        请安装 indentfirst 宏包,或联系模板维护者。\MessageBreak % chktex 1
    }
}

% 英语 / 欧文
\usepackage{fontspec}
\setmainfont{Times New Roman}
\setsansfont{Helvetica}
\setmonofont{Courier New}

% 日语 / 日本語
\setlength{\parindent}{10.5pt} % 设置段落缩进
\usepackage{indentfirst} 
\usepackage{xeCJK} 
\punctstyle{quanjiao}
%草稿: 进阶写法 \usepackage[AutoFakeBold=true, AutoFakeSlant=true]{xeCJK}
\xeCJKsetup{AutoFallBack=true}
\setCJKmainfont[BoldFont=YuMincho Demibold]{YuMincho Medium}
\setCJKsansfont{Hiragino Sans W5}
\setCJKmonofont{Hiragino Sans W3}

% 中文 / 中国語
\setCJKfallbackfamilyfont{\CJKrmdefault}{Songti SC} 
\setCJKfallbackfamilyfont{\CJKsfdefault}{Heiti SC} 
\setCJKfallbackfamilyfont{\CJKttdefault}{PingFang SC}
% 定义中文字体族
\setCJKfamilyfont{CNfont_songti}[BoldFont=Songti SC Black]{Songti SC}
\setCJKfamilyfont{CNfont_heiti}{Heiti SC}
\setCJKfamilyfont{CNfont_pingfangti}{PingFang SC}
\NewDocumentEnvironment{Chinese}{}{%
    \xeCJKsetup{AutoFallBack=true}%
    \CJKfamily{CNfont_songti}
}{}
% endregion
 % 语言配置
% updated 2026-01-03
% Math configuration 数学配置 数学の設定

% region ==== 数学字体设置 ==== 数学フォント設定 ====
% 基于 mathtools 和 unicode-math 宏包
% 实现数学符号和字体的配置
% 数学記号とフォントの設定を実現    
%—— 数学扩展宏包 —— 数学拡張パッケージ —— 
% mathtools: 数学扩展宏包, 提供更多数学符号和环境 数学拡張パッケージ, より多くの数学記号と環境を提供
% unicode-math: 支持 Unicode 数学字体的宏包 Unicode 数学フォントをサポートするパッケージ
% setmathfont: 设置数学字体
% Symbol: 学校指定 Symbol 字体 用于数字和直立希腊字母 学校指定の Symbol フォントを使用して数字と直立ギリシャ文字を表示
%—— 物理学符号宏包 —— 物理学記号パッケージ —— 
% physics: 提供物理学相关的符号和命令 物理学に関連する記号とコマンドを提供  
%依赖检查
% 检查宏包是否存在
\IfFileExists{mathtools.sty}{%
    \typeout{ mathtools 宏包已安装}%
}{%
    \PackageError{宏包缺失}{%
         mathtools.sty 未找到%
    }{%
        请安装 mathtools 宏包,或联系模板维护者。\MessageBreak % chktex 1
    }
}
\IfFileExists{unicode-math.sty}{%
    \typeout{ unicode-math 宏包已安装}%
}{%
    \PackageError{宏包缺失}{%
         unicode-math.sty 未找到%
    }{%
        请安装 unicode-math 宏包,或联系模板维护者。\MessageBreak % chktex 1
    }
}
\IfFileExists{physics.sty}{%
    \typeout{ physics 宏包已安装}%
}{%
    \PackageError{宏包缺失}{%
         physics.sty 未找到%
    }{%
        请安装 physics 宏包,或联系模板维护者。\MessageBreak % chktex 1
    }
}
\usepackage{mathtools} % 数学扩展宏包 要在 unicode-math 之前加载
\usepackage{unicode-math}
\setmathfont{NewComputerModernMath-Regular}  % 主数学字体
\setmathfont{Symbol}[ % 学校指定Symbol字体用于直立希腊字母
    range=up/{greek,Greek},       % 直立希腊字母
    Scale=MatchLowercase          % 保持大小匹配
]
\renewcommand{\familydefault}{\rmdefault}  % 强制使用衬线字体
\usepackage{physics} % 物理学符号宏包
% endregion
      % 数学配置
% updated 2026-01-03
% Table configuration 表格配置 表の設定
%=== 表格支持 === 表サポート ===
%依赖检查
% 检查宏包是否存在
\IfFileExists{booktabs.sty}{%
    \typeout{ booktabs宏包已安装}%
}{%
    \PackageError{宏包缺失}{%
         booktabs.sty 未找到%
    }{%
        请安装 booktabs 宏包,或联系模板维护者。\MessageBreak % chktex 1
    }
}
\IfFileExists{threeparttable.sty}{%
    \typeout{ threeparttable宏包已安装}%
}{%
    \PackageError{宏包缺失}{%
         threeparttable.sty 未找到%
    }{%
        请安装 threeparttable 宏包,或联系模板维护者。\MessageBreak % chktex 1
    }
}
\IfFileExists{multirow.sty}{%
    \typeout{ multirow宏包已安装}%
}{%
    \PackageError{宏包缺失}{%
         multirow.sty 未找到%
    }{%
        请安装 multirow 宏包,或联系模板维护者。\MessageBreak % chktex 1
    }
}
\IfFileExists{longtable.sty}{%
    \typeout{ longtable宏包已安装}%
}{%
    \PackageError{宏包缺失}{%
         longtable.sty 未找到%
    }{%
        请安装 longtable 宏包,或联系模板维护者。\MessageBreak % chktex 1
    }
}
%加载宏包
\usepackage{booktabs} % 支持表格
\usepackage{threeparttable} % 支持表格注释
\usepackage{multirow} % 支持表格多行合并
\usepackage{longtable} % 支持跨页表格
\usepackage{array} % 支持自定义列宽和对齐方式     % 表格配置
\numberwithin{equation}{subsection} % 公式编号按小节重新编号
\numberwithin{table}{subsection}    % 表编号按小节重新编号

\usepackage{graphicx}
\usepackage{geometry}
\usepackage{caption}
\usepackage{titlesec}
\usepackage{booktabs}
\usepackage{xparse}
\usepackage{pdfpages}
\usepackage{xcolor}
\usepackage[svgnames]{xcolor} 
\usepackage{tikz}
\usetikzlibrary{shapes}
\usepackage{circuitikz}
\usepackage{pgfplots}

\usepackage{lipsum} % 用于生成示例文本
\usepackage{listings}
%region ===== 颜色设置 ===== カラー設定 =====
\usepackage[dvipsnames]{xcolor}
\definecolor{tusgreen}{RGB}{0, 153, 0}
\definecolor{mygold}{RGB}{255, 242, 213}
\definecolor{mydarkgreen}{RGB}{29, 58, 50}
\definecolor{myorange}{HTML}{FBB034}
\definecolor{DarkGreen}{RGB}{29, 58, 50}
%endregion
\usepackage{float}
\usepackage{cuted}
\usepackage{textgreek}
\usepackage[most]{tcolorbox}
% \usepackage{pdfcomment} % 用于添加 PDF 注释和提示 和 \hl冲突, 和hyperref冲突
\usepackage{comment}

% ———— 特效 ————
\usepackage{shadowtext}
\shadowoffset{16pt} % 阴影偏移
\shadowcolor{gray} % 阴影颜色

\pgfplotsset{compat=1.18} % 可选版本
\usepackage{enumitem}
\geometry{a4paper, left=2.5cm, right=2.5cm, top=2.5cm, bottom=2.5cm}
\usepackage[unicode,colorlinks=true,linkcolor=DarkGreen, urlcolor=DarkGreen,citecolor=DarkGreen]{hyperref}
\hypersetup{
    linktoc=all,        % 编号和标题都可点击
    CJKbookmarks=true   % 中文书签支持
}
\usepackage{fancyhdr}
\newcommand{\hs}{\hspace{0.8em}}  % 文本用 half-space
\newcommand{\hd}{\hspace{0.5em}}  % 文本用 half-space
\renewcommand{\contentsname}{{\CJKfontspec{FZDaBiaoSong-B06S}\fontsize{40}{45}\selectfont 目录}}

\NewDocumentCommand{\hl}{ O{yellow} m }{%高级命令
  \colorbox{#1}{#2}%
}

% 定义 VSCode 风格颜色
\definecolor{vscBackground}{RGB}{30,30,30}
\definecolor{vscPlain}{RGB}{220,220,220}
\definecolor{vscKeyword}{RGB}{86,156,214}
\definecolor{vscString}{RGB}{206,145,120}
\definecolor{vscComment}{RGB}{87,166,74}
\definecolor{gray}{RGB}{150,150,150}
\lstset{
  language=Python,
  backgroundcolor=\color{vscBackground},
  basicstyle=\ttfamily\footnotesize\color{vscPlain},
  keywordstyle=\color{vscKeyword}\bfseries,
  stringstyle=\color{vscString},
  commentstyle=\color{vscComment}\itshape,
  numberstyle=\tiny\color{gray},
  numbers=left,
  stepnumber=1,
  frame=single,
  breaklines=true,
  tabsize=4,
  showstringspaces=false,
  captionpos=b
}
\usepackage{minted}  % 需要 -shell-escape

% MonokaiPro 背景
\definecolor{monokai-bg}{RGB}{39,40,34}
\setminted{
    style=xcode,   % friendly / xcode / default 可以试试
    fontsize=\footnotesize,
    breaklines,
    linenos,
    numbersep=5pt,
    autogobble,
    frame=single
}

\renewcommand{\lstlistingname}{コード}

\newcommand{\eq}[1]{
    \begin{equation}
        \begin{split}
            #1 
        \end{split}
    \end{equation}
}
\newcommand{\eqx}[1]{
    \[
        \begin{split}
            #1
        \end{split}
    \]
}
\newcommand{\eqkk}[2]{
    \begin{tcolorbox}[colback=white!95!myorange, colframe=myorange!80!black, title=#2]
    \begin{align}
    #1
    \end{align}
    \end{tcolorbox}
}
\newcommand{\eqxkk}[2]{
    \begin{tcolorbox}[colback=white!95!myorange, colframe=myorange!80!black, title=#2]
    \begin{align*}
    #1
    \end{align*}
    \end{tcolorbox}
}

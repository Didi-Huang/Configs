% update: 2026-01-17
%\documentclass[aspectratio=43, 12pt]{beamer}

%region ===== 基础编码和引擎设置 ===== 基本的なエンジンとエンコード =====
\usepackage[utf8]{inputenc} % UTF-8 字符编码 --- 文字コード UTF-8 ---
\usepackage[T1]{fontenc}    % 字体编码 --- フォントエンコード ---
\usepackage{expl3}          % LaTeX3 编程环境 --- LaTeX3 プログラミング環境 ---
%endregion

% updated 2026-01-03
% Language and font configuration 语言和字体配置 言語とフォントの設定

% region ==== 字体设置 ==== フォント設定 ====
% 基于 fontspec 和 xeCJK 宏包
% 实现英语, 日语, 中文混排  
% 英語, 日本語, 中国語 の混在を実現
%
% —— 欧文字体 —— 英語フォント —— 
% 设置英文的主字体、无衬线字体和等宽字体
% mainfont: 主字体 (serif)
% sansfont: 无衬线字体 (sans-serif)
% monofont: 等宽字体 (monospace)
%—— 日本語字体 —— 日本語フォント —— 
% setlength: \parindent 设置段落缩进 段落冒頭のインデント
% indentfirst: 
%     开启section后第一段的段首空格 section後の最初の段落もインデント 
%     注意:正式论文不一定需要, 可根据需要选择是否使用 注意: 正式な論文では必ずしも必要ではないので, 必要に応じて使用するかどうかを選択可能
% package xeCJK: 支持中日韩字体 日中韓フォントサポート
% xeCJKsetup: auto fallback 优先加载日语, 没有的字体则自动回退到中文字体 多言語混在の自動フォントフォールバック機能を有効化
%—— 中文字体 —— 中国語フォント —— 
%—— 纯中文环境 ——


% 检查宏包是否存在模版
%\IfFileExists{ceshi.sty}{%
%    \typeout{ ceshi 宏包已安装}%
%}{%
%    \typeout{PackageError 指令测试}%
%    \PackageError{宏包缺失}{%
%         ceshi.sty 未找到%
%    }{%
%        请安装 ceshi 宏包,或联系模板维护者。\MessageBreak % chktex 1
%    }
%}

%依赖检查
% 检查宏包是否存在
\IfFileExists{fontspec.sty}{%
    \typeout{ fontspec 宏包已安装}%
}{%
    \PackageError{宏包缺失}{%
         fontspec.sty 未找到%
    }{%
        请安装 fontspec 宏包,或联系模板维护者。\MessageBreak % chktex 1
    }
}
\IfFileExists{xeCJK.sty}{%
    \typeout{ xeCJK 宏包已安装}%
}{%
    \PackageError{宏包缺失}{%
         xeCJK.sty 未找到%
    }{%
        请安装 xeCJK 宏包,或联系模板维护者。\MessageBreak % chktex 1
    }
}
\IfFileExists{indentfirst.sty}{%
    \typeout{ indentfirst 宏包已安装}%
}{%
    \PackageError{宏包缺失}{%
         indentfirst.sty 未找到%
    }{%
        请安装 indentfirst 宏包,或联系模板维护者。\MessageBreak % chktex 1
    }
}

% 英语 / 欧文
\usepackage{fontspec}
\setmainfont{Times New Roman}
\setsansfont{Helvetica}
\setmonofont{Courier New}

% 日语 / 日本語
\setlength{\parindent}{10.5pt} % 设置段落缩进
\usepackage{indentfirst} 
\usepackage{xeCJK} 
\punctstyle{quanjiao}
%草稿: 进阶写法 \usepackage[AutoFakeBold=true, AutoFakeSlant=true]{xeCJK}
\xeCJKsetup{AutoFallBack=true}
\setCJKmainfont[BoldFont=YuMincho Demibold]{YuMincho Medium}
\setCJKsansfont{Hiragino Sans W5}
\setCJKmonofont{Hiragino Sans W3}

% 中文 / 中国語
\setCJKfallbackfamilyfont{\CJKrmdefault}{Songti SC} 
\setCJKfallbackfamilyfont{\CJKsfdefault}{Heiti SC} 
\setCJKfallbackfamilyfont{\CJKttdefault}{PingFang SC}
% 定义中文字体族
\setCJKfamilyfont{CNfont_songti}[BoldFont=Songti SC Black]{Songti SC}
\setCJKfamilyfont{CNfont_heiti}{Heiti SC}
\setCJKfamilyfont{CNfont_pingfangti}{PingFang SC}
\NewDocumentEnvironment{Chinese}{}{%
    \xeCJKsetup{AutoFallBack=true}%
    \CJKfamily{CNfont_songti}
}{}
% endregion
  %语言配置

% updated 2026-01-03
% Math configuration 数学配置 数学の設定

% region ==== 数学字体设置 ==== 数学フォント設定 ====
% 基于 mathtools 和 unicode-math 宏包
% 实现数学符号和字体的配置
% 数学記号とフォントの設定を実現    
%—— 数学扩展宏包 —— 数学拡張パッケージ —— 
% mathtools: 数学扩展宏包, 提供更多数学符号和环境 数学拡張パッケージ, より多くの数学記号と環境を提供
% unicode-math: 支持 Unicode 数学字体的宏包 Unicode 数学フォントをサポートするパッケージ
% setmathfont: 设置数学字体
% Symbol: 学校指定 Symbol 字体 用于数字和直立希腊字母 学校指定の Symbol フォントを使用して数字と直立ギリシャ文字を表示
%—— 物理学符号宏包 —— 物理学記号パッケージ —— 
% physics: 提供物理学相关的符号和命令 物理学に関連する記号とコマンドを提供  
%依赖检查
% 检查宏包是否存在
\IfFileExists{mathtools.sty}{%
    \typeout{ mathtools 宏包已安装}%
}{%
    \PackageError{宏包缺失}{%
         mathtools.sty 未找到%
    }{%
        请安装 mathtools 宏包,或联系模板维护者。\MessageBreak % chktex 1
    }
}
\IfFileExists{unicode-math.sty}{%
    \typeout{ unicode-math 宏包已安装}%
}{%
    \PackageError{宏包缺失}{%
         unicode-math.sty 未找到%
    }{%
        请安装 unicode-math 宏包,或联系模板维护者。\MessageBreak % chktex 1
    }
}
\IfFileExists{physics.sty}{%
    \typeout{ physics 宏包已安装}%
}{%
    \PackageError{宏包缺失}{%
         physics.sty 未找到%
    }{%
        请安装 physics 宏包,或联系模板维护者。\MessageBreak % chktex 1
    }
}
\usepackage{mathtools} % 数学扩展宏包 要在 unicode-math 之前加载
\usepackage{unicode-math}
\setmathfont{NewComputerModernMath-Regular}  % 主数学字体
\setmathfont{Symbol}[ % 学校指定Symbol字体用于直立希腊字母
    range=up/{greek,Greek},       % 直立希腊字母
    Scale=MatchLowercase          % 保持大小匹配
]
\renewcommand{\familydefault}{\rmdefault}  % 强制使用衬线字体
\usepackage{physics} % 物理学符号宏包
% endregion
       %数学公式配置
%\setlength{\jot}{0.4ex} % 调整多行公式行间距 slide里面要缩小行距节省空间


%region ===== 颜色设置 ===== カラー設定 =====
\usepackage[dvipsnames]{xcolor}
\definecolor{tusgreen}{RGB}{0, 153, 0}
\definecolor{mygold}{RGB}{255, 242, 213}
\definecolor{darkgreen}{RGB}{29, 58, 50}
\definecolor{darkred}{RGB}{139, 0, 0}
%endregion


%region ===== 图片和绘图支持 ===== 画像と描画サポート =====
%---- Tikz 环境 ---- Tikzパッケージ ----
\usepackage{tikz}
\def\mathdefault#1{#1}
\usetikzlibrary{arrows.meta} % 使用 TikZ 的箭头库
\usetikzlibrary{positioning}
\usetikzlibrary{angles, quotes}
\usetikzlibrary{shapes, arrows, positioning}
\usepackage{circuitikz} 
\ctikzset{resistors/thickness=1}
\ctikzset{grounds/thickness=1}



% ---- 图片支持 ---- 画像サポート ----
\usepackage{graphicx} % 支持插入图片
\usepackage{pgf} % 支持 pgf 格式图片
\usepackage{array}
\usepackage{cancel} % 支持公式删除线
\usepackage{pgfplots} % 支持绘图
%endregion

%region ===== Beamer 设置 ===== Beamer設定 =====
%---- 基础設定 ---- 基本設定 ----
\pgfplotsset{compat=1.18} % 设置 pgfplots 的兼容性版本
\usetheme{Madrid} % 使用 Madrid 主题
%\useoutertheme{miniframes} % 使用 smoothbars 外观主题, 备用方案: miniframes

\usefonttheme{professionalfonts} %开启衬线字体
\usefonttheme{serif} %开启衬线字体
\setbeamercolor{structure}{fg=darkgreen}
\setbeamercolor{frametitle}{fg=white, bg=darkgreen}
\setbeamercolor{item}{fg=darkgreen}
\setbeamercolor{block title}{bg=darkgreen, fg=white}
\setbeamercolor{block body}{bg=white, fg=black}
\setbeamercolor{title}{fg=white, bg=darkgreen}


%---- 页脚设置 ---- フッター設定 ----
%\setbeamertemplate{footline}[frame number] % 仅显示页码, 不要作者和标题
\setbeamertemplate{frametitle continuation}{} % 取消续页标题
\setbeamerfont{footline}{size=\scriptsize} % 页脚字体大小设为正常大小

%---- 取消自动续页标题 ---- 自動継続フレームタイトルの無効化 ----
\usepackage{etoolbox}
\makeatletter
\patchcmd{\beamer@continueautobreak}{\frametitle}{\beamer@gobbleoptional}{}{\errmessage{failed to patch}}
\patchcmd{\beamer@continueautobreak}{\framesubtitle}{\beamer@gobbleoptional}{}{\errmessage{failed to patch}}
\makeatother

% 34pt = 12%
% ----- 设置字体大小 -----
% ----- 封面 表紙 -----
\setbeamerfont{title}{size=\fontsize{30}{36}\selectfont} % 封面主标题  タイトルページ
\setbeamerfont{author}{size=\fontsize{14}{17}\selectfont}     % 作者 著者
\setbeamerfont{institute}{size=\fontsize{10}{12}\selectfont}  % 机构  所属
\setbeamerfont{date}{size=\fontsize{11}{13}\selectfont}       % 日期  日付

% ----- 目录 目次 -----
\setbeamerfont{section in toc}{size=\large} % 目录章节标题 セクション
\setbeamerfont{subsection in toc}{size=\normalsize} % 目录小节标题 サブセクション
\setbeamerfont{subsubsection in toc}{size=\normalsize} % 目录子小节标题 サブサブセクション 

% ----- 正文 本文 -----
\setbeamerfont{frametitle}{size=\fontsize{22}{26}\selectfont}   % 帧标题:22pt (65%) 
\setbeamerfont{normal text}{size=\fontsize{12}{14.4}\selectfont}  % 正文:12pt (35%)
\setbeamerfont{itemize/enumerate body}{size=\fontsize{16}{19}\selectfont} % 列表:16pt
\setbeamerfont{itemize/enumerate subbody}{size=\fontsize{14}{17}\selectfont} % 子列表:14pt
\setbeamerfont{block title}{size=\fontsize{18}{22}\selectfont}  % 块标题:18pt (53%)
\setbeamerfont{block body}{size=\fontsize{15}{18}\selectfont}   % 块内容:15pt (44%)

\usepackage{tikz} % 支持 TikZ 绘图
\usetikzlibrary{calc}
\def\mathdefault#1{#1}


\usepackage{multicol} % 支持多栏布局
\usepackage{hyperref} % 支持超链接
\pdfstringdefDisableCommands{
  \def\\{ }  % 在PDF书签中将\\替换为空格
  \def\hs{ } % 在PDF书签中忽略\hs
}
\usepackage{textcomp}
\usepackage{booktabs} % 支持表格
\usepackage{caption} % 支持图表标题
\usepackage[normalem]{ulem} % 支持下划线和删除线
\usepackage{fancyhdr} % 支持页眉页脚
\usepackage{lipsum} % 用于生成示例文本
\usepackage{booktabs}  % 标准表格绘制
\usepackage{etoolbox}
\makeatletter
\patchcmd{\beamer@sectionintoc}
  {\ifnum\c@tocdepth>0\relax}
  {\ifnum\c@tocdepth>2\relax}{}{}
\makeatother


\usepackage{annotate-equations} % 支持公式注释

\usepackage{comment}
\usepackage{hyperref}
\usepackage{multicol}% 支持多栏布局
\usepackage{verbatim} % 支持代码块
\usepackage{tabularx}
\renewcommand{\arraystretch}{1.5}
\newcommand{\hs}{\hspace{0.8em}}  
\newcommand{\hd}{\hspace{0.5em}}  
\NewDocumentCommand{\myhl}{ O{yellow} m }{%高级命令
  \colorbox{#1}{#2}%
}
%标题:7.1%宽 实际: 7.5%宽
%式子:5%宽 实际:5.5%宽
%正文:3.5%宽 实际:3.7%宽
%下角标:1.6%宽


%---- 标题页设置 ---- タイトルページ設定 ----
\newcommand{\mytoday}{$\the\year{}$ 年 $\the\month{}$ 月$\the\day{}$ 日}
\title[電子回路基礎実験IV]{物理学実験報告:\\ 電子回路基礎実験IV}%标题 タイトル
\author[黄\&下田]{$1224038$\ 黄\ 宏迪 \and $1224059$\ 下田\ 凜歩\\ 指導教員: ⾧田\ 朋\ 助教\vspace*{-0.5em} }% []只显示姓即可特定到个体因此简写 []には名字だけ表示して個人を特定できるように省略
\institute[]{東京理科大学 理学部第I部物理学科$2$年 A組\vspace*{-0.5em}} % 添加机构信息 []什么都不写以实现页脚不显示机构信息 []には何も書かないとフッターに所属が表示されない
\date[]{\mytoday} % 添加日期 []什么都不写以实现页脚不显示日期 []には何も書かないとフッターに日付が表示されない
%endregion

%region ===== 图表标题设置 ===== 図表キャプション設定 =====
%---- 图表标题设置 ---- 図表キャプション設定 ----
\DeclareCaptionFont{ninept}{\fontsize{8}{8}\selectfont}%图表标题字体大小为9pt
\captionsetup[figure]
{
  name=Figure,%or name=Figure,
  labelsep=period,
  font= ninept,
  labelfont={bf},
  textfont={normalfont}
}
\captionsetup[table]
{
  name=Table,%or name=Table,
  labelsep=period,
  font= ninept,
  labelfont={bf},
  textfont={normalfont}
}
%endregion

% region ===== 自定义命令 ===== カスタムコマンド =====
%---- 插入图片命令 ---- 画像挿入コマンド ----
\newcommand{\insertpgf}[5][0.5]{
    \begin{figure}[H]
        #5
        \centering
%        {\pgfplotsset{#6}}
        \resizebox{#1\textwidth}{!}{\input{#2}}
        \caption*{#3}\label{#4}
    \end{figure}
}%用法:\insertpgf{発表/実験1.pgf}{実験1:各波長の$I$-$V$特性}{fig:exp1}{\vspace{-3em}}
\newcommand{\insertpng}[5][0.5]{
    \begin{figure}[H]
        #5
        \centering
        \includegraphics[width=#1\textwidth]{#2}
        \caption*{#3}\label{#4}
    \end{figure}
}

\newcommand{\insertpdf}[5][0.5]{
    \begin{figure}[H]
        #5
        \centering
        \includegraphics[width=#1\textwidth]{#2}
        \caption{#3}\label{#4}
    \end{figure}
}

\newcommand{\insertpair}[6][0.45]{%
    \begin{figure}[H]
        \centering
        \begin{minipage}{0.49\textwidth}
            \centering
            \pgfimg[#1]{#2}{#3}
        \end{minipage}\hfill
        \begin{minipage}{0.49\textwidth}
            \centering
            \pgfimg[#1]{#4}{#5}
        \end{minipage}
        \caption{#6}\label{fig:pair}
    \end{figure}
}
\newcommand{\NaturalElementTextFormat}[4]
{
  \begin{minipage}{2.3cm}
    \centering
      \textbf{#1}
      \linebreak \linebreak
      {\Huge \textbf{{#3}}}
      \linebreak
       {#4}
      \linebreak
      {\small {#2}} 
  \end{minipage}
}
% endregion

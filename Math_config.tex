% updated 2026-01-03
% Math configuration 数学配置 数学の設定

% region ==== 数学字体设置 ==== 数学フォント設定 ====
% 基于 mathtools 和 unicode-math 宏包
% 实现数学符号和字体的配置
% 数学記号とフォントの設定を実現    
%—— 数学扩展宏包 —— 数学拡張パッケージ —— 
% mathtools: 数学扩展宏包, 提供更多数学符号和环境 数学拡張パッケージ, より多くの数学記号と環境を提供
% unicode-math: 支持 Unicode 数学字体的宏包 Unicode 数学フォントをサポートするパッケージ
% setmathfont: 设置数学字体
% Symbol: 学校指定 Symbol 字体 用于数字和直立希腊字母 学校指定の Symbol フォントを使用して数字と直立ギリシャ文字を表示
%—— 物理学符号宏包 —— 物理学記号パッケージ —— 
% physics: 提供物理学相关的符号和命令 物理学に関連する記号とコマンドを提供  
%依赖检查
% 检查宏包是否存在
\IfFileExists{mathtools.sty}{%
    \typeout{ mathtools 宏包已安装}%
}{%
    \PackageError{宏包缺失}{%
         mathtools.sty 未找到%
    }{%
        请安装 mathtools 宏包,或联系模板维护者。\MessageBreak % chktex 1
    }
}
\IfFileExists{unicode-math.sty}{%
    \typeout{ unicode-math 宏包已安装}%
}{%
    \PackageError{宏包缺失}{%
         unicode-math.sty 未找到%
    }{%
        请安装 unicode-math 宏包,或联系模板维护者。\MessageBreak % chktex 1
    }
}
\IfFileExists{physics.sty}{%
    \typeout{ physics 宏包已安装}%
}{%
    \PackageError{宏包缺失}{%
         physics.sty 未找到%
    }{%
        请安装 physics 宏包,或联系模板维护者。\MessageBreak % chktex 1
    }
}
\usepackage{mathtools} % 数学扩展宏包 要在 unicode-math 之前加载
\usepackage{unicode-math}
\setmathfont{NewComputerModernMath-Regular}  % 主数学字体
\setmathfont{Symbol}[ % 学校指定Symbol字体用于直立希腊字母
    range=up/{greek,Greek},       % 直立希腊字母
    Scale=MatchLowercase          % 保持大小匹配
]
\renewcommand{\familydefault}{\rmdefault}  % 强制使用衬线字体
\usepackage{physics} % 物理学符号宏包
% endregion
